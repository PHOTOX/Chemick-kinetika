\usepackage[czech]{babel}
\usepackage[version=3]{mhchem} % sázení chemických vzorců

\usepackage[ paperheight  =292mm,paperwidth   =204mm,  % or: "paper=a4paper"
             layoutheight =226 mm,layoutwidth  =148mm,
             layoutvoffset= 33mm,layouthoffset= 28mm,
             margin=0pt,
             showframe=false, showcrop=false]{geometry}

%%%%%% Nastaveni obrazku a tabulek %%%%%%%
\usepackage{graphicx}
\usepackage{epstopdf} %vkládání eps obrázků do pdf souboru

% Nastaveni tabulek
\usepackage{booktabs} %lepsi vzhled tabulek
%kontrola vdov a sirotku
\usepackage[defaultlines=4,all]{nowidow}
%lepsi pozicovani
\usepackage{float}

\usepackage[bf]{caption}   %tucne "Figure" a "table'
\captionsetup{width=0.9\textwidth} %úprava šířky popisků
\captionsetup{textfont=it, font=small} %úprava fontů u obrázků, text bude menší a italikou
\footskip 22pt
%\renewcommand{\thefootnote}{\fnsymbol{footnote}}  %footnote jako symboly
%\renewcommand{\thefootnote}{\alph{footnote}} %footnote jako mala pismenka

%%%%%%% Matematika a jiné
\usepackage[squaren]{SIunits} %spravne sazeni jednotek
\usepackage{amsfonts}
\usepackage{mathtools}
\usepackage{amssymb}  %matematické fonty
\usepackage{amsmath}  %umoznuje prikaz \numberwithin


% Toto je zkopirovano ze skript Kvantove chemie
% Je mozno odkomentovat dle potřeby
%\usepackage{rotating} % rotace tabulky
%\usepackage{multicol} % slouceni sloupcu v tabulce
%\usepackage{multirow} % slouceni radku v tabulce
%\usepackage{epigraph}% citaty
%\usepackage{mathrsfs} % fonty do teorie group 
%\usepackage{tcolorbox} % fancy staff
%\usepackage{lscape}
%\usepackage{mathbbol}
%\DeclareSymbolFontAlphabet{\amsmathbb}{AMSb}%
%\usepackage{picinpar}
%\usepackage{braket}


\usepackage{xcolor}
% here is a nice documentation for tcolorbox
%https://www.overleaf.com/latex/examples/drawing-coloured-boxes-using-colorbox/pvknncpjyfbp#.VmjpHx8SprQ1
\usepackage{tcolorbox}
\tcbuselibrary{breakable} %boxes over more pages

\usepackage{url}   %pro snadnou tvorbu URL adres

\usepackage[pdftitle={Kinetická chemie},
pdfauthor={P. Slavíček},
bookmarks=true,
colorlinks=true,
breaklinks=true,
urlcolor=red,
citecolor=blue,
linkcolor=blue,
unicode=true,
pdfstartview=FitV]{hyperref}


%%%%%% Definice vlastních věcí   %%%%%%%%%%%%%%

% redefinice vektoru, nyní značíme tučně
\renewcommand{\vec}{\mathbf}

% prikaz na vykresleni diferencialu \mathrm{d}
%\newcommand{\dd}{\mathrm{d}}
\newcommand\dd{\mathop{}\!\mathrm{d}}  % from http://tex.stackexchange.com/questions/60545/should-i-mathrm-the-d-in-my-integrals

% zkrácený výraz pro parciální derivaci
\newcommand{\pd}{\partial}

% zkraceny vyraz pro hmotnost elektronu
\newcommand{\me}{m_{\mathrm{e}}}

% zkraceny vyraz pro exponencialu
\newcommand{\ee}{\mathrm{e}}

%Kolafova zavorka pres 3 radky
\newcommand*\twobrace{$\displaystyle\left.\rule{0pt}{3.1ex}\right\}$}
\newcommand*\threebrace{$\displaystyle\left.\rule{0pt}{5ex}\right\}$}


\definecolor{ourgray}{gray}{0.8}
\colorlet{ourcyan}{cyan!10}
\newenvironment{ourbox}[1]
{\medskip
\begin{tcolorbox}[breakable,width=\textwidth,colback={ourcyan},title={\large #1},colbacktitle=ourgray,coltitle=black]
}
{\end{tcolorbox}}

%procedura definujici nove prostredi "priklad"
\definecolor{bleudefrance}{rgb}{0.19, 0.55, 0.91}
\newcounter{poradi} % novy citac pro cislo prikladu
\newcounter{CP} % novy citac pro cislo v labelu
\newcommand{\novepocitadlo}{\stepcounter{poradi}\theporadi} % realizace pocitadla pro cislo prikladu 
\newcommand{\novylable}[1]{\refstepcounter{CP}\label{#1}} % novy prikaz pro label, ktery resi spravny odkaz \ref
\newcommand{\novynadpis}{{\normalsize \sffamily \color{bleudefrance}{Příklad \novepocitadlo}} \vspace{0.1cm} \novylable{\theCP}} %prikaz, který vypise "Priklad xy"

\newenvironment{priklad}
{\vspace{0.2cm}\rule{0.93\textwidth}{1pt}
\small %\sffamily
\begin{tcolorbox}[breakable,colback=white,colbacktitle=white,coltitle=black,colframe=white]\novynadpis\\ \noindent}
{\end{tcolorbox}
\rule{0.93\textwidth}{1pt}\vspace{0.2cm}}

\numberwithin{equation}{section} %cislovani rovnic v kazde sekci zvlast
\graphicspath{{obrazky/}{../obrazky/}} %adresar pro obrazky

